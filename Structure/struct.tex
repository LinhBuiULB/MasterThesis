\documentclass{article}

\usepackage[utf8]{inputenc}
\usepackage[pdftex]{graphicx}
\usepackage[left=3cm,right=3cm,top=3cm,bottom=3cm]{geometry}
\usepackage[T1]{fontenc}
\usepackage[francais,english]{babel}
\newcommand*{\escape}[1]{\texttt{\textbackslash#1}}
\frenchbsetup{StandardLists=true}

\usepackage{amsmath}
\usepackage{amssymb}

\usepackage{listings}

%ALGORITHM
\usepackage{algorithm}
\usepackage[noend]{algpseudocode}
\renewcommand{\algorithmicforall}{\textbf{for each}}
\newcommand{\var}[1]{\mathit{#1}}
\newcommand{\func}[1]{\mathrm{#1}}
\algdef{SE}[DOWHILE]{Do}{doWhile}{\algorithmicdo}[1]{\algorithmicwhile\ #1}
%

\usepackage{caption}
\usepackage[hidelinks]{hyperref}
\usepackage{xcolor}
\usepackage{makecell}

\usepackage{graphicx}

\usepackage{fancyhdr}
\pagestyle{fancy}
\fancyhf{}
\fancyhead[R]{\thepage}


\title{Structure du Mémoire \\ Methods for large-scale image classification and application to biomedical data}
\author{\bsc{BUI QUANG PHUONG} Linh - ULB ID : 000427796 \\ MA1 Computer Sciences}
\date{March 2019}


\begin{document}

\maketitle

Comme indique le titre, le mémoire va être diviser en deux grandes parties qui vont constituer la lignée principale du sujet : la classification d'images classiques (pas de domaine particulier) et un focus plus précis sur des données biomédicales. \\

Quant à la structure du mémoire, en plus de la page de garde, du sommaire, du lexique, etc qui seront évidemment présents, le mémoire va se structurer comme tel :

\begin{enumerate}
\item \textbf{L'abstract} : résumé du problème et des méthodes abordés (à rédiger en fin de mémoire) 
\item \textbf{L'introduction} : 
\begin{itemize}
\item présentation du contexte (quand, comment ce problème est-il survenu) 
\item présentation des problèmes généraux de la vie courante relatifs à la classification d'images comme Google Images ou les caméras de surveillance + présentation d'exemples dans différents domaines notamment le domaine dans le domaine médical. 
\item Définition plus concrète du problème (assignation d'un label à une image) dont la sous-question : "Comment est donc représenté l'image?"
\item Brève présentation du machine learning et des méthodes associées (méthodes encore à déterminer) 
\item Présentation des objectifs 
\end{itemize}  

\item \textbf{Etat de l'art} : 
% Présentation du schéma classique de classification d'image et des différentes étapes tels que l'extraction de caractéristiques, le préprocessing, ...  

\end{enumerate} 



\end{document}