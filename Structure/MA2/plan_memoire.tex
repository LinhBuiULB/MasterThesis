\documentclass{article}

\usepackage[utf8]{inputenc}
\usepackage[pdftex]{graphicx}
\usepackage[left=3cm,right=3cm,top=3cm,bottom=3cm]{geometry}
\usepackage[T1]{fontenc}
\usepackage[francais,english]{babel}
\newcommand*{\escape}[1]{\texttt{\textbackslash#1}}
\frenchbsetup{StandardLists=true}

\usepackage{amsmath}
\usepackage{amssymb}

\usepackage{listings}

%ALGORITHM
\usepackage{algorithm}
\usepackage[noend]{algpseudocode}
\renewcommand{\algorithmicforall}{\textbf{for each}}
\newcommand{\var}[1]{\mathit{#1}}
\newcommand{\func}[1]{\mathrm{#1}}
\algdef{SE}[DOWHILE]{Do}{doWhile}{\algorithmicdo}[1]{\algorithmicwhile\ #1}
%

\usepackage{caption}
\usepackage[hidelinks]{hyperref}
\usepackage{xcolor}
\usepackage{makecell}

\usepackage{graphicx}

\usepackage{fancyhdr}
\pagestyle{fancy}
\fancyhf{}
\fancyhead[R]{\thepage}


\title{Structure du Mémoire \\ Methods for large-scale image classification and application to biomedical data}
\author{\bsc{BUI QUANG PHUONG} Quang Linh - ULB ID : 000427796 \\ MA2 Computer Sciences}
\date{Novembre 2019}


\begin{document}

\maketitle

Ce qui a été réalisé en \textbf{Master 1} : 

\begin{enumerate}
\item \textbf{Introduction} : présentation des concepts, historique, des exemples d'application. 

\item \textbf{General concepts of AI} : présentation du machine learning et deep learning, supervised/unsupervised learning, définition du problème de classification d'images. 

\item \textbf{Material and Methods} : présentation des neural network, CNN, SVM ainsi que des méthodes d'évaluation et de validation. 

\item \textbf{Breast cancer detection} : présentation d'un cas connu à partir d'un article, présentation de ses méthodes et résultats 

\end{enumerate}

\noindent\hrulefill

\vspace{0.5cm}
Ce qui va être réalisé en \textbf{Master 2} :

\begin{enumerate}
\setcounter{enumi}{4}

\item \textbf{Model prototype} : création d'un premier modèle et application sur des données non-médicales (CIFAR10). Présentation du dataset CIFAR10, présentation du modèle, des différents paramètres: 

\begin{itemize}
\item Tester différents optimisers : adam, SGD, ...
\item Tester avec et sans dropout + test de différentes valeurs de dropout
\item Tester avec weight decay (kernel\_regularizer)
\item Tester avec data augmentation
\item Tester différentes valeurs de hyperparameters (epochs, batch\_size, ...) 
\end{itemize}

$\rightarrow$ Manipuler et comparer/présenter les résultats obtenus en fonction de ces différentes valeurs et voir quelle configuration obtient les meilleurs résultats. + justification en expliquant les différents paramètres / variables.    

\item \textbf{A specific case: Identification of Leukemia Subtypes} (ou autre article/problème): présentation du problème en question, essayer notre modèle sur le dataset de l'article, voir si les paramètres optimisés sont les mêmes qu'avec un dataset classique (CIFAR10), appliquer les différentes méthodes/filtres utilisés (data augmentation) dans l'article afin de reproduire (voir améliorer) les résultats si possible. 

\item \textbf{Exclusive data analysis} : création d'un modèle pour des données exclusives, présentation des données et des résultats obtenus. (encore un peu flou, à compléter)  

\item \textbf{Interactive image labelling} : progresser sur l’annotation automatique d’images (active learning) (encore un peu flou, à compléter)  
% Chapitre 4: https://tel.archives-ouvertes.fr/tel-00752022v2/document 

\item \textbf{Combining different types of input data} : combinaison de différents types d'input lors de la création du modèle $\rightarrow$ Mixed Data (encore un peu flou, à compléter)   

% https://www.pyimagesearch.com/2019/02/04/keras-multiple-inputs-and-mixed-data/ 

\end{enumerate}


\textbf{N.B:} L'ordre des sections peut être modifié au cours du travail. \\

A première vue, les points 5 ainsi qu'une partie (voir totalement) du point 6 vont être réalisés pendant le premier quadri tandis que les 3 derniers points seront réservés au second quadri. 

\end{document}