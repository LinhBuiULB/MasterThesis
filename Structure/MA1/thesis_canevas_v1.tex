\documentclass{article}

\usepackage[utf8]{inputenc}
\usepackage[pdftex]{graphicx}
\usepackage[left=3cm,right=3cm,top=3cm,bottom=3cm]{geometry}
\usepackage[T1]{fontenc}
\usepackage[francais,english]{babel}
\newcommand*{\escape}[1]{\texttt{\textbackslash#1}}
\frenchbsetup{StandardLists=true}

\usepackage{amsmath}
\usepackage{amssymb}

\usepackage{listings}

%ALGORITHM
\usepackage{algorithm}
\usepackage[noend]{algpseudocode}
\renewcommand{\algorithmicforall}{\textbf{for each}}
\newcommand{\var}[1]{\mathit{#1}}
\newcommand{\func}[1]{\mathrm{#1}}
\algdef{SE}[DOWHILE]{Do}{doWhile}{\algorithmicdo}[1]{\algorithmicwhile\ #1}
%

\usepackage{caption}
\usepackage[hidelinks]{hyperref}
\usepackage{xcolor}
\usepackage{makecell}

\usepackage{graphicx}

\usepackage{fancyhdr}
\pagestyle{fancy}
\fancyhf{}
\fancyhead[R]{\thepage}


\title{Structure du Mémoire \\ Methods for large-scale image classification and application to biomedical data}
\author{\bsc{BUI QUANG PHUONG} Quang Linh - ULB ID : 000427796 \\ MA1 Computer Sciences}
\date{March 2019}


\begin{document}

\maketitle

Comme indique le titre, le mémoire va être diviser en deux grandes parties qui vont constituer la lignée principale du sujet : la classification d'images classiques (pas de domaine particulier) et un focus plus précis sur des données biomédicales. \\

Quant à la structure du mémoire, en plus de la page de garde, du sommaire, du lexique, etc qui seront évidemment présents, le début du mémoire va se structurer comme tel , donc en \textit{ce qui concerne cette année} (MA1) : 

\begin{enumerate}
\item \textbf{L'abstract} : résumé du problème et des méthodes abordés (à rédiger en fin de mémoire) 
\item \textbf{L'introduction} : 
\begin{itemize}
\item présentation du contexte (quand, comment ce problème est-il survenu) 
\item présentation des problèmes généraux de la vie courante relatifs à la classification d'images comme Google Images ou les caméras de surveillance + présentation d'exemples dans différents domaines notamment le domaine dans le domaine médical (détection de tumeur / radiographie / etc). 
\item Définition plus concrète du problème (assignation d'un label à une image) dont la sous-question : "Comment est donc représenté l'image?" 
\item Présentation des objectifs 
\end{itemize}  

\item \textbf{Background} : 
\begin{itemize}
\item Introduction du machine learning / deep learning et des méthodes associées - par exemple Convolutional Neural Networks, SIFT, SURF, SVM, etc (méthodes encore à déterminer)
\item Introduction à la réduction de dimension et de visualisation
\end{itemize}

\item \textbf{Etat de l'art} : 
Dans un premier temps, présentation des méthodes classiques/générales pour la classification d'images : 
\begin{itemize}
\item Présentation du schéma classique de classification d'image et des différentes étapes tels que l'extraction de caractéristiques, le préprocessing, ...  
\item Comparaison des méthodes abordées, quels sont les spécificités de chaque méthode ? 
\item Présentation de quelques résultats obtenus par les différentes références bibliographiques 
\item Avantages et limites du Machine Learning / Deep Learning (et donc de ces méthodes) ?
\item Efficacité ? Performance ? Précision des résultats ?
\end{itemize}

Puis dans un second temps, on va s'intéresser aux données biomédicales : 
\begin{itemize}
\item Quels méthodes sont utilisées généralement pour ce genre de données ? 
\item Présentation de quelques résultats sur des données des références bibliographiques. 
\item Tout comme pour les images classiques, efficacité ? Performance ? Précision des résultats ?
\item Etude de cas sur un certain sous-problème, par exemple détection d'une lésion, ...
\end{itemize}

\item (Section provisoire pour la MA1) \textbf{Objectifs et plan pour la suite du mémoire} : 
\begin{itemize}
\item présentation des projets à venir pour la MA2, des sujets et méthodes abordées, des données qui vont être traitées, etc. 
\end{itemize}

\end{enumerate} 

\end{document}